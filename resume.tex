\documentclass[margin,line]{resume}
\usepackage[defblank]{paralist}
\usepackage{pdfpages}
\usepackage{anysize}
\usepackage[pdftex, unicode]{hyperref}
\hypersetup{
	pdftitle={Zachary Hart's Resume},
	pdfauthor={Zachary Hart},
	pdfborder={0 0 0},
	% Not sure if this is needed.
	unicode=true
}

\marginsize{0.75in}{2.0in}{0.5in}{0.5in}
\setdefaultitem{\footnotesize \textbullet}{}{}{}{}{}
\setdefaultleftmargin{0em}{}{}{}{}{}
\setdefaultenum{(a)}{(1)}{}{}{}{}
\newcommand{\rurl}[1]{\hfill {\footnotesize \url{#1}}}
\newcommand{\rdate}[1]{\hfill {\small #1}}
\renewcommand{\employer}[5]{\item[#1] - #2 \rdate{#3} \\* #4 \rurl{#5} \\*}
\begin{document}
\name{\Huge Zachary Hart}
\begin{resume}
\section{\mysidestyle Contact Information} 
	\begin{asparablank}
		\item 2109 Becket Drive  \hfill \href{mailto:zach@csh.rit.edu}{zach@csh.rit.edu}
		\item Flower Mound, Texas 75028 \hfill (972) 786-5794
		\item \hfill \rurl{github.com/zthart}
    \end{asparablank}

\section{\mysidestyle Objective}
	\begin{asparablank}
    	\item Seeking a co-op in software engineering for the Summer of 2019.
        \normalsize
        \\
	\end{asparablank}

\section{\mysidestyle Education}
	\begin{compactdesc}
		\item[Rochester Institute of Technology] - Rochester, NY \rdate{August 2015 - Present}
		\begin{compactitem} { \small
			\item Major: B.S. Computer Science
			\item Expected graduation: June 2020 (5-year program)
            \item Relevant Coursework: Data Structures \& Algorithms, Mechanics of Programming, Intro to Software Engineering, Concepts of Computer Systems, Concepts of Parallel and Distributed Systems, Programming Language Concepts, Principles of Data Management
		} \end{compactitem}
	\end{compactdesc}

\section{\mysidestyle Technical Skills}
	\begin{compactdesc}
		\item[Languages] \begin{inparaenum} { \small
			Python, C, Rust, Java, MIPS Assembly
		} \end{inparaenum}
        \item[Frameworks \& Libraries] \begin{inparaenum} { \small
        	Flask, Django, Requests, Pytest
        } \end{inparaenum}
		\item[Operating Systems] \begin{inparaenum} { \small
			Linux (CentOS/RHEL/Fedora, Debian/Ubuntu), Windows, macOS Sierra
		} \end{inparaenum}
        \item[Technical Interests] \begin{inparaenum} { \small
			6502 Assembly, General Digital Electronics, Hardware Prototyping, Home Automation
        } \end{inparaenum}
        \normalsize
	\end{compactdesc}

\section{\mysidestyle Work Experience}
	\begin{asparablank}
		\item{\bf Syncurity}\rurl{https://syncurity.net}
		\small \item Integrations Engineer \hfill July 2017 - Present
		\linebreak

		\small Collaborated with a remote team in a small and fast-paced startup environment to integrate third party APIs and services into our product using \textbf{Python}.
		\small Wrote integrations that have been certified by leading companies in the security space (\textbf{McAfee}, \textbf{Splunk}), and participated in partner programs with many more.
		\small Started development on a new integrations framework/\textbf{RESTful API} using \textbf{Flask}, \textbf{RabbitMQ}, \textbf{Vault}, and other technologies for job dispatching and management.
		\small Gained experience in working with a team of engineers and security analysts both within the company and as clients, remote and on-site, to understand the needs of our users.
	\end{asparablank}

\section{\mysidestyle Projects}
	\begin{asparablank}
		\item {\bf Pourover}\rurl{https://github.com/zthart/pourover}

		\small A \textbf{Python} library for parsing and manipulating CEF messages and log files - used in production in an enterprise environment, and available on pypi.
		\normalsize
		\\
		\item {\bf CSH Lounge Automation}\rurl{https://github.com/zthart/csh-automation}

		\small Aggregates control of consumer devices such as A/V receivers, projectors, and lighting control and implements \textbf{X10 Automation}, \textbf{HDMI-CEC}, and serial communications to interface with the devices. Exposes a RESTful API in order to allow control from any internet enabled device.
		\normalsize
        \\
		\item {\bf CSH Compute Cluster}\rurl{https://github.com/zthart/csh-slurmguide}

		\small Setup and Management of a small computing cluster created for members of the Computer Science House at RIT. Allows for users to submit jobs to be run and managed using \textbf{SLURM Workload Manager}.
		\\
		\item {\bf Huffman Lite}\rurl{https://github.com/zthart/huffman-lite}

		\small Written in \textbf{C}, Huffman Lite is a huffman-like encoding and compression algorithm. Utilizing a tree structure similar to a huffman tree, Huffman Lite can encode and decode both human readible files and binary files.
		\normalsize
	\end{asparablank}
    
\section{\mysidestyle Extracurricular}
	\begin{asparablank}
		\item {\bf CSH ({\small Computer Science House})}\rurl{http://www.csh.rit.edu/}
		\small	\item House Improvements Director \hfill August 2016 - June 2017
        \small  \item Drink Administrator \hfill March 2016 - Present
        \small  \item Member \hfill August 2015 - Present
        \linebreak
        
        \small Computer Science House, or CSH, is a group of technically-minded students at the Rochester Institute of Technology that all share a goal of learning and creating through collaboration. Members past and present have been responsible for creating many projects that have seen success at RIT, as well as in the public space.
	\end{asparablank}
    

\end{resume}
\end{document}
